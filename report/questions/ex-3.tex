\section*{Exercise 3}
In this exercise, we show an example that the cyclic permutations are not enough to estimate the Jaccard similarity correctly.

Let's take a look at the example in table \ref{example}. We will perform a cyclic permutation to estimate the Jaccard similarity.

\begin{table}[h]
  \centering
\caption{Example}
\label{example}
\begin{tabular}{@{}lll@{}}
  \toprule
  & S1 & S2 \\ \midrule
a & 1  & 0  \\
b & 1  & 1  \\
c & 0  & 0  \\
d & 0  & 1  \\
e & 1  & 1  \\ \bottomrule
\end{tabular}
\end{table}

In this example, we can see that the 2 sets have 2 common words, ``b'' and ``e''. In general, $S1 \cap S2 = \{b,e\}$ and $S1 \cup S2 = \{a,b,d,e\}$. Thus the (correct) Jaccard similarity is $|S1 \cap S2| / |S1 \cup S2| = 1/2$.

Now we perform the cyclic permutation. 

\begin{itemize}
    \item a-b-c-d-e: The signature is 1 - 2.
    \item b-c-d-e-a: The signature is 1 - 1.
    \item c-d-e-a-b: The signature is 3 - 2.
    \item d-e-a-b-c: The signature is 2 - 1.
    \item e-a-b-c-d: The signature is 1 - 1.
\end{itemize}

We have 2 rows which contain the same values on 2 cells, and 5 rows in total. So the estimated Jaccard similarity is $2/5$, which is not correct. The reason is the $0 - 0$ row, which makes the next rows to be counted twice in the intersection (while every row is counted only once in the total number of permutations).

