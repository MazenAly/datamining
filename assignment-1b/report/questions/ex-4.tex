\section*{Exercise 4}
\subsection*{a}
In this exercise, we consider another variant of the clustering problem, the
k-median problem. The idea is similar to k-means in the sense that the distance
from each point to its center is minimized.

The k-median approach can be more robust against noise because noise does not 
affect the order of the data, so the medians are not modified if the noise 
increases. If we use k-means, noise can biase the centroids. 

\subsection*{b}
We use a similar algorithm to k-means, just replace the means by the medians. Thatis, when updating a cluster, we take the point having the medians of the features as coordinates as the new center. This approach is obviously not optimal in term of cost-minimization, as the solution is proven in Exercise 2. So we will 
experiment the actual cost we get in the set C3.txt.
